\subsection{Requerimientos Funcionales}

\subsubsection{Generación Procedural de Niveles}
\begin{itemize}
    \item El sistema debe generar un nivel único y dinámico cada vez que el jugador comience una nueva partida.
    \item La generación debe incluir la disposición de enemigos, obstáculos y otros elementos del entorno.
\end{itemize}

\subsubsection{Ajuste Dinámico de Dificultad}
\begin{itemize}
    \item La inteligencia artificial debe ajustar la dificultad del juego en tiempo real, en función del rendimiento del jugador durante la partida.
    \item El ajuste de la dificultad debe tener en cuenta variables como la precisión, número de vidas perdidas, y puntuación obtenida.
\end{itemize}

\subsubsection{Algoritmo de Aprendizaje Automático}
\begin{itemize}
    \item El sistema debe recopilar datos de las últimas 15 partidas del jugador para entrenar el modelo de IA.
    \item El modelo debe ser capaz de evolucionar según las mejoras o retrocesos en las habilidades del jugador.
\end{itemize}

\subsubsection{Sistema de Feedback Visual y Sonoro}
\begin{itemize}
    \item El sistema debe proporcionar feedback visual y sonoro para indicar cambios en la dificultad (por ejemplo, un aumento de velocidad en los enemigos o más poder de fuego en el jugador).
\end{itemize}

\subsubsection{Múltiples Tipos de Enemigos y Patrón de Ataques}
\begin{itemize}
    \item El juego debe contar con una variedad de enemigos con diferentes patrones de ataque, que se adapten en dificultad según la experiencia del jugador.
    \item Los patrones de ataque de los enemigos se deben ajustar a correspondencia del nivel de dificultad y tipo de enemigo.
    %\item Los patrones de ataque también deben poder ajustarse de manera procedural en función de los niveles generados.
\end{itemize}

\subsubsection{Sistema de Puntuación y Métricas}
\begin{itemize}
    \item El juego debe registrar estadísticas detalladas (puntuación, enemigos eliminados, tiempo jugado, etc.) que puedan ser utilizadas para ajustar la dificultad.
    \item Debe mostrar estas estadísticas al jugador al final de cada partida.
\end{itemize}

\subsubsection{Manejo de Guardado de Datos}
\begin{itemize}
    \item El sistema debe ser capaz de guardar y cargar datos del progreso del jugador, así como las configuraciones de IA para partidas futuras.
\end{itemize}

\subsection{Requerimientos No Funcionales}

\subsubsection{Rendimiento}
\begin{itemize}
    \item El juego debe funcionar de manera fluida, con un frame rate mínimo de 60 FPS en condiciones estándar.
\end{itemize}

\subsubsection{Escalabilidad del Algoritmo de IA}
\begin{itemize}
    \item Debe permitir la integración de nuevas variables de juego sin necesidad de rediseñar el sistema.
\end{itemize}

\subsubsection{Usabilidad}
\begin{itemize}
    \item El juego debe ofrecer una interfaz intuitiva para que los jugadores de cualquier nivel puedan entender los controles y mecánicas del juego.
    \item El sistema de ajuste de dificultad debe ser transparente para el jugador, sin necesidad de intervención manual.
\end{itemize}
