\subsubsection{Vista  Funcional}
Este diagrama representa la Vista de Caso de Uso para la Generación Procedural de Niveles. Describe las interacciones entre los actores principales del sistema y el proceso de generación de niveles.
\begin{figure}[H]
\centering
\begin{adjustbox}{width=0.45\textwidth}
\begin{tikzpicture}[
    process/.style={rectangle, draw, minimum height=1cm, minimum width=3.5cm, align=center},
    entity/.style={ellipse, draw, minimum height=1cm, minimum width=2.5cm, align=center},
    relation/.style={-Stealth, thick}
]

% Entities
\node[entity] (jugador) {Jugador};
\node[entity, below=of jugador] (partida) {Partida};
\node[entity, right=of partida] (nivel) {Nivel};
\node[entity, below=of partida] (enemigo) {Enemigo};
\node[entity, right=of enemigo] (ataque) {Ataque};

% Processes
\node[process, right=of jugador, xshift=3cm] (vars) {Recepción de Variables de Dificultad};
\node[process, below=of vars] (enemigoSel) {Selección de Enemigos};
\node[process, below=of enemigoSel] (patronGen) {Generación de Patrones de Ataque};
\node[process, below=of patronGen] (ajuste) {Ajuste Procedural de Patrones};
\node[process, below=of ajuste] (verif) {Verificación de Reglas de Juego};
\node[process, below=of verif] (implement) {Implementación de Enemigos en el Nivel};

% Arrows between processes
\draw[relation] (vars) -- (enemigoSel);
\draw[relation] (enemigoSel) -- (patronGen);
\draw[relation] (patronGen) -- (ajuste);
\draw[relation] (ajuste) -- (verif);
\draw[relation] (verif) -- (implement);

% Arrows between entities and processes
\draw[relation] (jugador) -- (vars);
\draw[relation] (nivel) -- (enemigoSel);
\draw[relation] (enemigo) -- (enemigoSel);
\draw[relation] (ataque) -- (patronGen);

% Relationships
\draw[relation, dashed] (jugador) -- (partida) node[midway, right] {Juega};
\draw[relation, dashed] (partida) -- (nivel) node[midway, below] {Posee};
\draw[relation, dashed] (enemigo) -- (ataque) node[midway, below] {Dispara};

\end{tikzpicture}
\end{adjustbox}
\caption{Vista Funcional}
\end{figure}
